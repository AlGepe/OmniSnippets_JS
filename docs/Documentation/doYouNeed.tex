\subsection{You only need food, water and sleep}
\label{sub:need}
In the beginning there was Mateusz. He started making calculators at speeds never known to humankind before. But he couldn't make them perfect for he was only (partially) human. This meant prioritizing speed and throughput over quality and visuals. This was good in the pre-calculatorian times at Omni.

However, with time, Omni started to produce a rare breed of humans, humans with the ability of empowering other humans to calculate anything and, impossible as it seemed at the time, do so while having fun. These highly evolved calculatorians grew in size quickly allowing for priorities to be shifted from throughput to quality; Omni didn't need to just exist, it needed to be awesome, the best.

You see, when Omni was created customJS was nothing more than a time saver, a workaround, an 'unlocking' feature. However, with the advent of the calculatorians-age customJS is being more and more widely use in calculators, which is reflected in more beautiful calculators, more user friendly tools and overall better quality products. 

It is for this reason that writting customJS has gone from a dark art to a truly useful skill that every calculatorian should know. From adding images, to making interactive calculators that show/hide variables depending on the user input, to graphs or even having different calculators in one, customJS can take your calculator from good to awesome!

That said, you shouldn't just use customJS for the sake of it. Omni calculators have to be as simple and easy to use as possible (without compromising functionality), so some times it is better to stick to the standard procedure than sacrifice user experience. 

To assess better if your calculator could benefit from some customJS goodness, the best way is to look at some examples and understand what is possible, what is recommended and what is not recommended or just plain impossible to do; so let's do that!

\subsection{Do what you can because you must}
\label{sub:whenToCJS}
We will now show you the functionalities that customJS provides and its (dis)advantages by looking at 3 different scenarios. One where customJS is not needed or useful, one where customJS is not needed but can add extra functionality (so customJS would be "desirable") and one where customJS is needed.

\subsubsection{CustomJS is uncalled for}
\label{subsub:unneeded}
Example: \href{https://www.omnicalculator.com/all/percentage-app} {Simple Percentage}
Conveniently, this example is actually Omni's \#first ever calculator and shows clearly when CustomJS might be an unnecessary addition. This is a very simple calculator that most people will use to quickly calculate the value they need. In this case we want to keep calculators as simple and easy to use as possible. 

Another important reason why customJS is probably not a good addition is the fact that the concepts calculated and shown are simple enough that they don't require any visual aid, and there is also not clear visualization of the topics at hand that would greatly improve the usability or explanations that accompany the calculations. 

In cases like this, one should just make a simple calculator and focus on making a kick-ass text to go with it where the user can learn more about the topic. 

With all this said, is important to note that it is never cut-and-dry, a pie chart would help the user visualise what a given percentage means. It could be a decent addition depending on who visits the webpage, nevertheless we really think that in this particular case your time is be better invested elsewhere. 

\textbf{More Examples of unneeded CustomJS}
\begin{itemize}
    \item \href{https://www.omnicalculator.com/math/percentage-increase}{Percentage Increase Calculator}
    \item \href{https://www.omnicalculator.com/math/percentage-of-percentage}{Percentage of a Percentage Calculator}
    \item \href{URL}{Calculator}
\end{itemize}

\subsubsection{CustomJS is desirable}
\label{subsub:desireable}
Example: \href{https://www.omnicalculator.com/math/pythagorean-theorem}{Pythogorean Theorem}
This is also a very simple calculator, that doesn't need to include customJS to be a decent calculator. 

However, a simple picture makes everything much easier to understand and simple to explain. You don't need to add help text to the variables and you can use the standard mathematical notation for each of the sides of the triangle. Which is a better approach than using "First/Second Cathetus" and "Hypotenuse" which assume some pre-existing. 

A picture and some added text with the equation take this calculator from "usable" or "decent" to "very good" and that's something we want to do as often as we can. 

This is by far the simplest example of customJS usage, but exemplifies very well the benefits of adding some customJS in the right places.  Some other, more complex but still useful additions of customJS can be seen here. Note that none of this calculators "require" the use of customJS but greatly benefit from it.

\textbf{More Examples of desireable CustomJS}
\begin{itemize}
    \item \href{url}{Calc}
\end{itemize}

\subsubsection{CustomJS is compulsory}
\label{subsub:compulsory}

And now we get to the interesting part of this section: the calculators for which you must use customJS. These fall in one of three categories:
\begin{itemize}
    \item Calculators that need pre-sets
    \item Calculators with non-standard calculations
    \item Calculators suffering from \href{https://en.wikipedia.org/wiki/Dissociative_identity_disorder}{Dissociative identity disorder}
\end{itemize}

\textbf{Pre-sets}\\
\label{calc:preset}
An example of a calculator that requires pre-sets is the \href{https://www.omnicalculator.com/food/chocolate}{Chocolate calculator} [id:941]. In this calculator there are predefined values that are not variables and are only shown in a table. These values can only be edited using customJS and are the main output of the calculator, making it compulsory to use customJS. Another, less standard example of a calculator using customJS as a way to offer pre-sets is the \href{https://www.omnicalculator.com/all/bike-size}{Bike Size Calculator}[id:1629] where \textit{type of bike} changes slightly the behaviour of the calculator. As a third and final example we have the most standard of all, which is the \href{https://www.omnicalculator.com/all/screen-size}{Screen Size Calculator}[id:832] the screen size is controlled by a simple \texttt{omni.valueSetter}. 

We haven't yet talked about them, but the customJS function commonly used in this type of calculator is the \texttt{omni.valueSetter}. We will talk about it in more detail in Section \ref{subsub:vSetter} but for now it's enough to know it give the user the option to changes the values of multiple variables at the same time via Calculatorian-defined tables.

\textbf{Weird formulas}\\
\label{calc:weird}
These calculators are those for which the calculation requires formulas or procedures that cannot be implemented via the \textit{Equations} tab on the calculator editor. The simplest example of such a calculator is the \href{https://www.omnicalculator.com/all/factorial}{Factorial Calculator} [id:395] that requires the input to be integer before it can output the result. Other examples along the same lines are:
\begin{itemize}
    \item \href{https://www.omnicalculator.com/all/prime-factorization}{Prime Factorization Calculator} [id:143]
    \item \href{https://www.omnicalculator.com/all/gcf-and-lcm}{GCF and LCM Calculator} [id:171]
\end{itemize}

You can clearly see a common theme where the calculation processe requires some extra steps that are not provided by any standard function that you can simple input in the calculator editor. Once again, we will see in more detail what kind of equations and formulas we actually have access to when we talk about Javascript (Section \ref{sec:fundamentals}) and the functions available at Omni (Chapter \ref{chap:omni})

\textbf{Multiple personality disorder}\\
\label{calc:multiple}
Some times when you make a calculator you want to add different options and behaviours so that you effectively have many calculator in one and the user simply changes between them by selecting from a drop-down menu. This kind of multiple personalty calculators are not always the best options but we all know that SEO works in mysterious ways, so at times is the best option, just make sure to confirm before you build.

Examples of these types of calculators are:
\begin{itemize}
    \item \href{https://www.omnicalculator.com/all/distance}{Distance Calculator} [id:144]
    \item \href{https://www.omnicalculator.com/all/area}{Area Calculator} [id:1569]
\end{itemize}

There is also a fourth type of calculator that is feared by calculatorians for its tough requirements, weird calculations and compulsory, unavoidable use of customJS: the marketing calculators. We will talk about them in the following section

\subsection{when freedom is subjugated to the marketing needs}
\label{sub:marketing}
There are times in life when one needs must surrender its own needs and desires for a greater good. For calculatorians this time has been given the name of \textbf{\textit{Marketing Calculators}} . And just as a good soldier must follow orders even against their own interest, a calculatorian must follow the guidance of the Marketing Team when the time comes. 

In all seriousness, though, marketing calculators are a special breeding for which many of the ordinary rules and guidelines bust be ignored or at least relegated to a secondary role. Marketing calculators have different goals than regular calculators and therefore their requirements are different. One of these differences is the fact that customJS is a must. 

In these instances customJs is not necessarily used to improve the functionality of the calculator but, mainly, for improving the user experience and fun-factor of the calculator, some times actively reducing the capabilities of the calculator since the main aim of it is not to solve a problem but to engage and entertain the user. 

There are countless examples of these calculators but have curated some of the (subjectively\footnote{Possibly unrelated fact: all these examples were created by the author of this document}) best marketing calculators of all time:
\begin{itemize}
    \item \href{https://www.omnicalculator.com/all/chilled-drink}{Chilled Drink Calculator} [id:1556]
    \item \href{https://www.omnicalculator.com/all/christmas-tree}{Christmas Tree Calculator} [id:1240]
    \item \href{https://www.omnicalculator.com/all/exoplanet}{Exoplanet Discovery Calculator} [id:1825]
\end{itemize}

All these calculators tend to use customJs for some or all the reasons listed above, but also for reasons directly related to user experience. For example, the Chill Drink Calculator hides most of the technical variables behind user-selectable options that are easy to understand by laypeople. The Christmas tree calculator also does that and adds interactive graphics that help the user preview the results of the calculations. The exoplanet calculator is a perfect example of breaking almost every single rule about creating calculators heavily including html code to help the user understand and visualise the results.

    For marketing calculators the \large{\textbf{tl;dr}} is that calculators should be entertaining and simple, they should include customJS and EVERYTHING is overridden by whatever the Natalia says at the time of building the calculator.
