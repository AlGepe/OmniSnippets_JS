\documentclass[11pt,a4paper,oldfontcommands]{memoir}
\usepackage[utf8]{inputenc}
\usepackage[T1]{fontenc}
\usepackage{microtype}
\usepackage[dvips]{graphicx}
\usepackage{xcolor}
\usepackage{times}

\usepackage[
breaklinks=true,colorlinks=true,
%linkcolor=blue,urlcolor=blue,citecolor=blue,% PDF VIEW
linkcolor=black,urlcolor=black,citecolor=black,% PRINT
bookmarks=true,bookmarksopenlevel=2]{hyperref}

\usepackage{geometry}
% PDF VIEW
% \geometry{total={210mm,297mm},
% left=25mm,right=25mm,%
% bindingoffset=0mm, top=25mm,bottom=25mm}
% PRINT
\geometry{total={210mm,297mm},
left=20mm,right=20mm,
bindingoffset=10mm, top=25mm,bottom=25mm}

\OnehalfSpacing
%\linespread{1.3}

%%% CHAPTER'S STYLE
\chapterstyle{bianchi}
%\chapterstyle{ger}
%\chapterstyle{madsen}
%\chapterstyle{ell}
%%% STYLE OF SECTIONS, SUBSECTIONS, AND SUBSUBSECTIONS
\setsecheadstyle{\Large\bfseries\sffamily\raggedright}
\setsubsecheadstyle{\large\bfseries\sffamily\raggedright}
\setsubsubsecheadstyle{\bfseries\sffamily\raggedright}


%%% STYLE OF PAGES NUMBERING
%\pagestyle{companion}\nouppercaseheads 
%\pagestyle{headings}
%\pagestyle{Ruled}
\pagestyle{plain}
\makepagestyle{plain}
\makeevenfoot{plain}{\thepage}{}{}
\makeoddfoot{plain}{}{}{\thepage}
\makeevenhead{plain}{}{}{}
\makeoddhead{plain}{}{}{}


\maxsecnumdepth{subsection} % chapters, sections, and subsections are numbered
\maxtocdepth{subsection} % chapters, sections, and subsections are in the Table of Contents

\begin{document}

%%%---%%%---%%%---%%%---%%%---%%%---%%%---%%%---%%%---%%%---%%%---%%%---%%%
%   TITLEPAGE
%
%   due to variety of titlepage schemes it is probably better to make titlepage manually
%
%%%---%%%---%%%---%%%---%%%---%%%---%%%---%%%---%%%---%%%---%%%---%%%---%%%
\thispagestyle{empty}

{%%%

\centering
\title{ \normalsize \textsc{}
		\\ [2.0cm]
		\hline \vspace{10pt} \\
		\huge \textbf{CustomJS primer for Calculatorians\textsuperscript{\textregistered}} \vspace{5pt}
		\textbf{\hline}{\hline} \\ [0.5cm]
		\large \vspace{200pt}\\}		}
		

%\date{\today}

\author{\LARGE Alvaro Diez \\
		Not an Astrophysicist\texttrademark \\
		\LARGE Dominik Czernia \\
		Mr. CustomJS Wizzard himself \\ \\ \\ \\ \\ \\ \\
		\LARGE {\textbf{Omni Calculator Project}} \vspace{-50pt}
} 
		
\maketitle
%%%
\noalign
\clearpage

\tableofcontents

\clearpage

%%%---%%%---%%%---%%%---%%%---%%%---%%%---%%%---%%%---%%%---%%%---%%%---%%%
%%%---%%%---%%%---%%%---%%%---%%%---%%%---%%%---%%%---%%%---%%%---%%%---%%%

\chapter{Before you start}

Dzien dobry lub dobry wiczor. Before we get to business, it's worth taking a look at what this \ttt{Primer} is and what it isn't, so that we don't waste your time.
This \textit{CustomJS primer} is meant as a quick-reference, or star-guide or user-manual for Calculatorians to first get started using Javascript in their calculators and later solve quick simple doubts that you might have while using that knowledge.

This is NOT a formal, technical, precise, dense, (hopefully) boring, all-encompassing\footnote{Thanks for the word Jack} Javascript book or written lecture on programming. This document aims to provide understandable, applicable knowledge and will sacrifice technicality and precision if needed. 

If you have any programming experience you will find the first sections of the second chapter to be extremely basic and you might prefer to start reading from the section \ref{}{}. If you already know how to program in Javascript, you might find most of it useless and some even probably offensive (if you style choices are worse than ours) so we recommend to just read those sections that relate to the application of such knowledge to making OmniCalculators such as REFSSS

Before we move on with the actual content let's faff around just a bit more and take a look at the different parts of this document and what to expect from them:

\begin{itemize}
    \item Chapter 1.- \textbf{Before you start} 
        \subitem Section 1.1 - A brief description of that is CustomJS what it does and how and when to use it
        \subitem Section 1.2 - Quick overview of the functions specific to omni that don't exist in regular JS
\subitem Section 1.3 - An \textit{Gepe-complete}\footnote{\textit{Gepe-complete} referes to the information that is fully complete to Al's eyes} start guide explaining the basic of general programming and the specifics of JS used at Omni
        \subitem Section 1.4 - A collection of \textit{rules} that we have in place at Omni and their reasons to exist
    \item Chapter 2.- \textbf{Okay so you are already coding...}
        \subitem Section 2.1 - Basic and advance use cases in Custom JS with strong focus on Omni functions
        \subitem Section 2.2 - How to prevent errors and what to do when they happen
        \subitem Section 2.3 - Style guide as agreed by most of Omni Calculatorians. Follow or be bullied.

\section{Who is this CustomJS guy?\small[An introduction to customJS]}
\subsection{You only need food, water and sleep}
\label{sub:need}
Most calculators don't require customJS

\subsection{Do what you can because you must}
\label{sub:whenToCJS}
But CJS is generally a good addition to most calcualtors (pictures, interactive options...)

\subsection{when freedom is subjugated to the marketing needs}
\label{sub:marketing}
If you're doing a marketing calculator you MUST please the marketing team, and they will not settle for anything without some customJS in it. Make it wacky!



    


\section{CustomJS at Omni \small{[Built-in functions]}}
\subsection{onInit}
and its functions (and shortcomings) here
\subsection{onResult}
and its many more funtions here


\section{Programming vs witchcraft spells \small[Fundamentals of coding]}
\subsection{What is a program?} 
\label{sub:program}
Text that tells the computer what to (only here for completeness)

\subsection{Javascript vs HTML}
\label{sub:jsHtml}
JS does
HTML shows

\subsection{Variables, functions, operations...}
\label{sub:types}
    \subsubsection{Variables - Primitives}
    \label{subsub:primitives}
int vs float. Number vs string. boolean
    \subsubsection{Operations}
    \label{subsub:operations}
obv, isn't it? but it depends on the type of variable
    \subsubsection{Variables - Compounding like I have interest} 
    \label{subsub:array}
Arrays, lists, dictionaries, objects...
    \subsubsection{Functions}
    \label{subsub:functions}
Interactive variables

\subsection{Order of execution and loops - Basics}
\label{sub:execBasic}
Bla bla bla up to down unless modifiers or functions.
    \subsubsection{if (if-else)}
    \label{subsub:if}
don't over use them

    \subsubsection{for}
    \label{subsub:for}
the protytpe loop

    \subsubsection{while}
    \label{sub:while}
for's brother

    \subsubsection{break}
    \label{subsub:break}
DENIED!

    \subsubsection{switch...case}
    \label{subsub:switch}
A fancy if, technically faster, only use for clarity

\subsection{Order of execution and loops - Advanced}
\label{sub:execAdv}
Don't use, but they are cool, so maybe use?
    \subsubsection{do-while}
    \label{subsub:doWhile}
for's weird cousin

    \subsubsection{labeled}
    \label{subsub:labeled}
Make it your own!

    \subsubsection{continue}
    \label{subsub:continue}
if you need help: \href{http://letmegooglethat.com/?q=continue}{click here}

    \subsubsection{for...in}
    \label{subsub:forIn}
for's weird cousing from Alabama

    \subsubsection{for...of}
    \label{subsub:forOf}
for's weird-cousin-from-Alabama's normal son
    
\subsection{The laziness principle}
\label{sub:lazy}
If it takes more than 5min to do think if someone might have done it before and look for it (or ask politely)
If you're doing the same thing more than 3 times, it can probably be automated. Never write the same thing (or almost the same thing) more than 5 times, there's surely a more efficient way\footnote{Exceptions might apply}


\section{A short list of strong suggestions \small{[Do's and Don'ts ]}}
\subsection{Do's}
\label{sub:dos}
follow the rules
\subsection{Don'ts}
\label{sub:donts}
follow the rules ALWAYS

\begin{itemize}
    \item
\end{itemize}


\chapter{Okay, so you are already coding...}

\section{What can you do \small{[typical additions in customJS]}}
\subsection{The obvious answer}
omni.functions \&\& his friend 
\subsection{Mix it up, spice it up!}
you are free\footnote{restrictions may apply. Free will is not guaranteed by Omni or the Universe}
\subsection{Useful examples}
source from trello
\subsection{How to memorize everything}
Don't!

\begin{itemize}
    \item
\end{itemize}


\section{Sh*t! Why is this not working! \small{[debugging for normies]}}
\subsection{The disappearing calculator}
You made a mistake, find it
\subsection{The error message in place of the calculator}
You made a mistake, find it using this hints (google = friend)
\subsection{The "everything works but the result is wrong"}
Maybe you've triggered some unexpected behaviour, check how you are using omni functions and ctx functions
\subsection{developer options, call html for help and other tricks}
plan ahead and your life will be easier. Then again, where is the fun in that?




\section{Do yourself a favor, do your colleagues a favor \small{[Style guide]}}
\subsection{I don't like rules, why do we have them?}

\subsection{Organizing the code}
Omni.defines
functions
general variables
onInit
OnResult
[final convention to be determined by democratic voting cause idc enough to be a dictator]
\subsection{Formal style conventions}
Bracket positioning, indentations, truncation of lines, spaces...
[final convention to be determined by democratic voting cause idc enough to be a dictator]
\subsubsection{Naming conventions}
thisIsAVariableNameThatLooksGood\newline
this\_i\_do\_not\_like\_but\_is\_alright\_i\_guess\newline
this\_ISHorrendous\newline
DontDoThis\newline
andneitherdothispls\newline
\subsubsection{Commenting}
Comment the weird bits, comment for visual aid
comment as much as needed and as little as possible
\subsubsection{Space vs Tabs: the age old debate nobody should've had :(\_)}
Tabs are for losers, end of the story




\section{Okay, but how can I...? \small{[additional resource]}}


\appendix

\chapter{To infinity and beyond!}
\section{A collection of helpful resources}
\input{resources.tex}

\bibliographystyle{unsrt}
\bibliography{sample}

\end{document}

