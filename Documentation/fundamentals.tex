\subsection{What is a program?} 
\label{sub:program}
We promised these will be a very brief and simple overview, so let us deliver just that:

A program is a text file that contains precise instructions understandable by a computer. The computer will perform this instructions in order as they are written (from top to bottom). 

For a computer to understand a program you need to follow a certain convention, which is often called the syntax, and that is specific to each programming language. Here we will only talk about the javascript syntax (and briefly html). Bare in mind that even if the syntax is language-dependent, most of the concepts are the same across different programming languages\footnote{This is almost always the case when we are talking about the most basic concepts such as the ones we will mention in this document}.

For us calculatorians javascript is the main language we will be using, but sometimes we might come across some \textit{HTML}, you need not worry about it, since it's recommended not to use it directly, but for completeness we will take a look at what it is and how it compares to javascript.
\subsection{Javascript vs HTML}
\label{sub:jsHtml}
Picture this: You are happily looking at someone's calculator looking for inspiration when suddenly weird characters appear on the screen and threaten you with their 'totally not javascript' looks, or how you call it now:\textit{syntax}. They probably look like this: \texttt{<a href='http...'>}. Fear not! This a friendly characters that have been relegated to the darkest parts of BB, but that love showing up here and there to help you when you've already lost all hope or javascript is simply not having a good day. 

HTML's are not the same spices as Javascript, though. But they are the perfect companion for javascript and a loyal friend that help your javascript actually be used by people. Let me explain:

Javascript is similar to general purpose programming languages like Python, C or Java\footnote{It's similar but DEFINITELY different} and hence it is used to perform calculations, run algorithms and make webpages dynamic. On the other hand HTML is closer to \LaTeX, or markdown (what we use to write the text of the calculators) and it's main purpose is to display things and make them look pretty. When people talk about HTML they usually mention also CSS which is (for our level, anyway) just an ordered way to create and store visual styles and desgin rules that HTML will follow, kind of like a configuration file for CSS.

There are many more differences between Javascript and HTML, like for example the fact that Javascript is generally ran on the user's computer (which is important to keep in mind when we build complex calculators) while HTML is ran on our server and only the results are shown in the user's computer. 

Compatibility is also an important factor to take into account when using Javascript and HTML. HTML tends to be much more widely supported across all browsers (nothing is perfect, though), while Javascript does present significant compatibility concerns for those who program websites from the ground up, which is not us\footnote{Sorry Darek and Daniel \^\^ <3}. Luckily we have nice developers that deal with those problems for us and don't allow us to use incompatible functions, which can be a bit frustrating when making very complex calculators, but it ensures that every user that visit our webpage will get a very nice calculator that works as intended.

\large{\textbf{tl;dr}}
Javascript is for doing and HTML is for showing it to people.
Javascript is run by the computer of the user, and our server does the HTML

\subsection{Variables, functions, operations...}
\label{sub:types}

And now into coding proper. Just a small side note before we dive into it: \textit{\textbf{Don't Panic!}} Programming does seem like witchcraft but unlike witchcraft it (1) really actually works and (2) if you make a mistake you don't actually risk the fate of the universe or anyone's life.

I once told my father \textit{"Don't you worry when using a computer, you don't know enough about it to break anything serious"} and exactly this philosophy applies to you: if youare reading this guide you probably don't have the knowledge to truly break anything. With that said, $don't be a dick$\texttrademark don't try to prove me wrong, don't do things that you are not sure about and if in doubt don't ask yourself "what would this do?", but rather "who can I ask about it?".  As long as you don't try to kill flies with bazookas\footnote{Spanish saying} you will be fine, and you should not worry at all.

\subsubsection{Variables - Primitives}
\label{subsub:primitives}
First of all we need to start with the idea of a variable. A variable is a reference, a name, a shorthand or nickname to refer to something in a much more clear and efficient way.  We do this in $Real Life$\texttrademark\xspace  when we talk to people. For example, you would never say: "super enthusiastic, very demaning, tall, hard working, really nice boss/colleague, and the reason any of us finish a difficult calculator \textit{on time}" and instead you would simply say \textbf{Bogna}, because it's simply quicker and conveys the same meaning.
We can do the same thing in any programming language by defining variables and assigning them some values or properties. In Javascript all variables are defined in the same way:
\begin{lstlisting}
        var Bogna = "super enthusiastic, very demaning, tall, hard working, really nice boss/colleague, and the reason any of us finish a difficult calculator on time"
\end{lstlisting}

In Javascript all variabels can be defined in the same way \texttt{var nameOfVariable = [expresion]} but it's important to understand the basic types of variables that exist, becuase each behaves differently and have different properties. Most variables cannot be mixed together and they require some kind of 'translation' from one type to another. Javascript does a good job to automatically convert but sometimes you might find weird errors popping up due to incompatible types being mixed up.
Here is a list of the most basic variable types:
\begin{itemize}
    \item \textbf{int}: Integer number
    \item \textbf{float}: Decimal number
    \item \textbf{char}: One letter or character
    \item \textbf{string}: List of characters or piece of text. Think about it as a sentence
    \item \textbf{bool}: Binary logical variable which can be set to True or False
\end{itemize}

You will always use \texttt{var} before the variable name to declare a variable (declaring means creating for us) independent of the type of variable. When declaring a string you need to put quotes at the begining and at the end of the text, but it doesn't matter if you use sing ($'$) or double quotes ($"$). It's important to keep in mind what type we're working on. Specially important is when we want to operate with them since depending on the types we are mixing we will get different results, or even crash errors. Let's see more about this in the next subsection.  

\large{\textbf{tl;dr}}
Variables are a nickname for any expression we want and help us write less and same more. Variables can have many types and we mostly never care about them except when we do.

\subsubsection{Operations}
\label{subsub:operations}

Operations are actions that you can perform with one or more variables to modify their value or create a new variable with some combination of the previous values. The most commonly used and notably simple are the mathematical operations such as \textbf{+},\textbf{-}, \textbf{*}(multiplication), \textbf{/}(division). Which should be self-evident. A more uncommon operator that is also very useful is the modulo operator (\textbf{\%}) which returns the remainder of the first variable devided by the second one. The usage is very simple: \texttt{remainder = numerator\%denominator}, and a practical example using number would be: \texttt{37\%5} which would return the value \texttt{2} since 37/5 = 7 with a remainder of 2, which is the output of the modulo function. 

Just remember that this is not an extensive documentation for javascript, so if you ever need any functionality or operation that we haven't mentioned, ask Google before you give up and look for a workaround. Speaking of things that are by no means exhaustively covered in this document: let's talk about operation on strings.

Strings are special variables in the sense that they act in many occasions as a single variable but can be also used as a collection of characters, kindof like a list of characters. The fact that the also carry information in natural language makes them more likely to be searched, splitted, slightly modified or even reformated. The gods of Javascript (yes, javascript is a revealed truth, not the creation of humans) though of these and gave us multiple operations that relate only to strings. The call them methods for reasons that will become apparent when we talk about objects in section \ref{subsub:array}.

We will mention here the (subjectively) most useful ones but make sure to check the reference for a complete list of all the methods available and a description of what they do and how to call them (that is, use them). Here we will only mention the three that we\ref{It's just me making this decisions but I want to sound not like a dictator} feel are the most useful. That said, in Omni Calculators strings are mostly just concatenated using the + operator, so you might never use any of these three.

\begin{itemize}
    \item \texttt{search()} Searches a string for a specified substring and returns the position of the match
    \item \texttt{replace()} Searches a string for a specified value and returns a new string where the specified values are replaced
    \item \texttt{split()} Splits a string into an array of substrings
    \item \texttt{length()} Returns the length of a string (number of characters)
\end{itemize}

An more complete list of all string methods can be found on \href{https://www.w3schools.com/jsref/jsref_obj_string.asp}{W3Schools} do check it out whenever you want to play around with strings, there are many options that can save you lots of time.

\large{textbf{tl;dr}}
Operations are sum (+), substract (-), multiply (*) and devide (/) but there are others we barely use. For strings no all this functions work but sum just add one string at the end of another. For all the string-specific operations, just look it up online (in the text above are some decent links). Just know that you can split, find characters, and other cool stuff.

\subsubsection{Variables - Compounding like I have interest} 
\label{subsub:array}
So this is all well and good, but what happens when you have to mange hundreds or thousands of variables that are very similar to each other, or very closely related? \xcancel{You\: give\: up.} You use compounded variables! (as I have taken the liberty of calling them for the sake of a pun)

many of you might have spent way too much time because you weren't taught about arrays and how to operate with them\footnote{I'm really sorry Julia, we could've save you so much time with the Pomodoro, hopefully this will still help you in the future}
Arrays, lists, dictionaries, objects...
\subsubsection{Functions}
\label{subsub:functions}
Interactive variables

\subsection{Order of execution and loops - Basics}
\label{sub:execBasic}
Bla bla bla up to down unless modifiers or functions.
\subsubsection{if (if-else)}
\label{subsub:if}
don't over use them

\subsubsection{for}
\label{subsub:for}
the protytpe loop \ref{sub:whoAsk}

\subsubsection{while}
\label{sub:while}
for's brother

\subsubsection{break}
\label{subsub:break}
DENIED!

\subsubsection{switch...case}
\label{subsub:switch}
A fancy if, technically faster, only use for clarity

\subsection{Order of execution and loops - Advanced}
\label{sub:execAdv}
Don't use, but they are cool, so maybe use?
\subsubsection{do-while}
\label{subsub:doWhile}
for's weird cousin

\subsubsection{labeled}
\label{subsub:labeled}
Make it your own!

\subsubsection{continue}
\label{subsub:continue}
if you need help: \href{http://letmegooglethat.com/?q=continue}{click here}

\subsubsection{for...in}
\label{subsub:forIn}
for's weird cousing from Alabama

\subsubsection{for...of}
\label{subsub:forOf}
for's weird-cousin-from-Alabama's normal son

\subsection{The laziness principle}
\label{sub:lazy}
If it takes more than 5min to do think if someone might have done it before and look for it (or ask politely)
If you're doing the same thing more than 3 times, it can probably be automated. Never write the same thing (or almost the same thing) more than 5 times, there's surely a more efficient way\footnote{Exceptions might apply}

\begin{itemize}
    \item
\end{itemize}
