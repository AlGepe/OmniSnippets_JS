\subsection{What is a program?} 
\label{sub:program}
We promised these will be a very brief and simple overview, so let us deliver just that:

A program is a text file that contains precise instructions understandable by a computer. The computer will perform this instructions in order as they are written (from top to bottom). 

For a computer to understand a program you need to follow a certain convention, which is often called the syntax, and that is specific to each programming language. Here we will only talk about the javascript syntax (and briefly html). Bare in mind that even if the syntax is language-dependent, most of the concepts are the same across different programming languages\footnote{This is almost always the case when we are talking about the most basic concepts such as the ones we will mention in this document}.

For us calculatorians javascript is the main language we will be using, but sometimes we might come across some \textit{HTML}, you need not worry about it, since it's recommended not to use it directly, but for completeness we will take a look at what it is and how it compares to javascript.
\subsection{Javascript vs HTML}
\label{sub:jsHtml}
Picture this: You are happily looking at someone's calculator looking for inspiration when suddenly weird characters appear on the screen and threaten you with their 'totally not javascript' looks, or how you call it now:\texit{syntax}. They probably look like this: \texttt{<a href='http...'>}. Fear not! This a friendly characters that have been relegated to the darkest parts of BB, but that love showing up here and there to help you when you've already lost all hope or javascript is simply not having a good day. 


JS does
HTML shows

\subsection{Variables, functions, operations...}
\label{sub:types}
    \subsubsection{Variables - Primitives}
    \label{subsub:primitives}
int vs float. Number vs string. boolean
    \subsubsection{Operations}
    \label{subsub:operations}
obv, isn't it? but it depends on the type of variable
    \subsubsection{Variables - Compounding like I have interest} 
    \label{subsub:array}
Arrays, lists, dictionaries, objects...
    \subsubsection{Functions}
    \label{subsub:functions}
Interactive variables

\subsection{Order of execution and loops - Basics}
\label{sub:execBasic}
Bla bla bla up to down unless modifiers or functions.
    \subsubsection{if (if-else)}
    \label{subsub:if}
don't over use them

    \subsubsection{for}
    \label{subsub:for}
the protytpe loop \ref{sub:whoAsk}

    \subsubsection{while}
    \label{sub:while}
for's brother

    \subsubsection{break}
    \label{subsub:break}
DENIED!

    \subsubsection{switch...case}
    \label{subsub:switch}
A fancy if, technically faster, only use for clarity

\subsection{Order of execution and loops - Advanced}
\label{sub:execAdv}
Don't use, but they are cool, so maybe use?
    \subsubsection{do-while}
    \label{subsub:doWhile}
for's weird cousin

    \subsubsection{labeled}
    \label{subsub:labeled}
Make it your own!

    \subsubsection{continue}
    \label{subsub:continue}
if you need help: \href{http://letmegooglethat.com/?q=continue}{click here}

    \subsubsection{for...in}
    \label{subsub:forIn}
for's weird cousing from Alabama

    \subsubsection{for...of}
    \label{subsub:forOf}
for's weird-cousin-from-Alabama's normal son
    
\subsection{The laziness principle}
\label{sub:lazy}
If it takes more than 5min to do think if someone might have done it before and look for it (or ask politely)
If you're doing the same thing more than 3 times, it can probably be automated. Never write the same thing (or almost the same thing) more than 5 times, there's surely a more efficient way\footnote{Exceptions might apply}
