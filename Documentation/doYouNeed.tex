\subsection{You only need food, water and sleep}
\label{sub:need}
In the beginning there was Mateusz. He started making calculators at speeds never known to humankind ever before. But he couldn't make them perfect for he was only (partially) human. This meant prioritizing speed and throughput over quality and visuals. This was good in the pre-calculatorian times at Omni.

However, with time, Omni started to haverst a rare breed of humans, humans with the ability of empowering other humans to calculate anything and, impossible as it seemed at the time, do so while having fun. These highly evolved calculatorians grew in size quickly allowing for priorities to be shifted from throughput to quality; Omni didn't need to just exist, it needed to be awesome, the best.

You see, when Omni was created customJS was nothing more than a time saver, a workaround, an 'unlocking' feature. However, with the advent of the calculatorians-age customJS is being more and more widely use in calculators, which is reflected in more beautiful calculators, more user friendly tools and overall better quality products. 

It is for this reason that writting customJS has gone from a dark art to a truly useful skill that every calculatorian should know. From adding images, to making interactive calculators that show/hide variables depending on the user input, to graphs or even having different calculators in one, customJS can take your calculator from good to awesome!

That said, you shouldn't just use customJS for the sake of it. Omni calculators have to be as simple and easy to use as possible (without compromising functionality), so some times it is better to stick to the standard procedure than sacrifice user experience. For example, a simple percentage calculator needs not be complicated with (dis)appearing fields, extra functionalities and complicated graphs.

To assess better if your calculator could benefit from some customJS goodness, the best way is to look at some examples and understand what is possible, what is recommended and what is not recommended or just plain impossible to do; so let's do that!

\subsection{Do what you can because you must}
\label{sub:whenToCJS}
We will now show you the functionalities that customJS provides and its (dis)advantages by looking at 3 different scenarios. One where customJS is not needed or useful, one where customJS is not needed but can add extra functionality (so customJS would be "desirable") and one where customJS is needed.

\subsubsection{CustomJS is uncalled for}
\label{subsub:unneeded}
Example: \href{Simple Percentage}{https://www.omnicalculator.com/all/percentage-app} 
Conveniently, this example is actually Omni's \#first ever calculator and shows clearly when CustomJS might be an unnecessary addition. This is a very simple calculator that most people will use to quickly calculate the value they need. In this case we want to keep calculators as simple and easy to use as possible. 

Another important reason why customJS is probably not a good addition is the fact that the concepts calculated and shown are simple enough that they don't require any visual aid, and there is also not clear visualization of the topics at hand that would greatly improve the usability or explanations that accompany the calculations. 

In cases like this, one should just make a simple calculator and focus on making a kick-ass text to go with it where the user can learn more about the topic. 

With all this said, is important to note that it is never cut-and-dry, a pie chart would help the user visualise the what a given percentage means. It could be a decent addition depending on who visits the webpage, nevertheless we really think that in this particular case your time would be better invested elsewhere. 

\textbf{More Examples of unneeded CustomJS}
\begin{itemize}
    \item \href{Percentage Increase Calculator}{https://www.omnicalculator.com/math/percentage-increase}
    \item \href{Percentage of a Percentage Calculator}{https://www.omnicalculator.com/math/percentage-of-percentage}
    \item \href{Calculator}{}
\end{itemize}

\subsubsection{CustomJS is desirable}
\label{subsub:desireable}
Example: \herf{Pythogorean Theorem}{https://www.omnicalculator.com/math/pythagorean-theorem}
This is also a very simple calculator, that doesn't need to include customJS to be a decent calculator. 

However, a simple picture makes everything much easier to understand and simple to explain. You don't need to add help text to the variables and you can use the standard mathematical notation for each of the sides of the triangle in stead of having to name them something more confusing like "First/Second Cathetus" and "Hypotenuse" which are names that, though understandable, assume some pre-existing. 

A picture and some added text with the equation take this calculator from "usable" or "decent" to "very good" and that's something we want to do as often as we can. This is by far the simplest example of customJS usage, but exemplifies very well the benefits of adding some customJS in the right places. 

Some other more complex but still useful additions of customJS can be seen here. None of this calculators require the use of customJS but greatly benefit from it.

\textbf{More Examples of desireable CustomJS}
\begin{itemize}
    \item \href{}{}
\end{itemize}

\subsubsection{CustomJS is compulsory}
\label{subsub:compulsory}

\textbf{More Examples of compulsory CustomJS}
\begin{itemize}
    \item \href{Percentage Increase Calculator}{https://www.omnicalculator.com/math/percentage-increase}
    \item \href{Percentage of a Percentage Calculator}{https://www.omnicalculator.com/math/percentage-of-percentage}
    \item \href{Calculator}{}
\end{itemize}
\subsection{when freedom is subjugated to the marketing needs}
\label{sub:marketing}
If you're doing a marketing calculator you MUST please the marketing team, and they will not settle for anything without some customJS in it. Make it wacky!


\textbf{More Examples of Marketing Calculators}
\begin{itemize}
    \item \href{Percentage Increase Calculator}{https://www.omnicalculator.com/math/percentage-increase}
    \item \href{Percentage of a Percentage Calculator}{https://www.omnicalculator.com/math/percentage-of-percentage}
    \item \href{Calculator}{}
\end{itemize}

    
